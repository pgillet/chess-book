\documentclass[a5paper]{article}

%-------------------------------------------------------------------------------
% PACKAGES
%-------------------------------------------------------------------------------

% Set page geometry: A5 paper with no margins so the background can fill the page.
\usepackage[a5paper, margin=0cm]{geometry}

% For including graphics.
\usepackage{graphicx}

% For advanced positioning of elements on the page.
% The 'overlay' and 'remember picture' options are essential for this task.
\usepackage{tikz}
\usetikzlibrary{calc} % Needed for coordinate calculations

% For proper handling of French characters and accents.
\usepackage[utf8]{inputenc}
\usepackage[T1]{fontenc}

%-------------------------------------------------------------------------------
% DOCUMENT
%-------------------------------------------------------------------------------

\begin{document}

% Remove page numbers and any other headers/footers.
\pagestyle{empty}

% Use a tikzpicture environment that covers the entire page to place all elements.
\begin{tikzpicture}[remember picture, overlay]

    % 1. BACKGROUND IMAGE
    % Place the background image node at the center of the page.
    % Make it fill the entire paper width and height.
    % 'inner sep=0' removes padding around the image.
    \node[at=(current page.center), inner sep=0pt] {
        \includegraphics[width=\paperwidth, height=\paperheight]{back-cover-image-3.png}
    };

    % 2. DEFINE THE SAFE ZONE
    % Define coordinates for the corners of the content area, respecting the 2cm margin.
    % This creates a "safe zone" inside which the text and logo will be placed.
    \coordinate (SafeZoneTopLeft)     at ($(current page.north west) + (2cm, -2cm)$);
    \coordinate (SafeZoneBottomRight) at ($(current page.south east) + (-2cm, 2cm)$);

    % Define coordinates for the top-center and bottom-center of the safe zone.
    % These will be used to vertically position the logo.
    \coordinate (SafeZoneTopCenter)   at ($(SafeZoneTopLeft)!.5!(SafeZoneTopLeft -| SafeZoneBottomRight)$);
    \coordinate (SafeZoneBottomCenter)at ($(SafeZoneBottomRight)!.5!(SafeZoneBottomRight -| SafeZoneTopLeft)$);

    % 3. PLACE THE TEXT BOX
    % Place a text node in the top-left corner of the safe zone.
    % 'anchor=north west' ensures the top-left of the box is at the specified coordinate.
    % 'align=left' respects the newlines (\\).
    % 'text=white' sets the color. You can change this (e.g., text=black).
    \node[
        anchor=north west,
        align=left,
        text=white,
        font=\sffamily % Using a sans-serif font, often looks better on covers
    ] at (SafeZoneTopLeft) {
        Un livre qui n'en est pas un.\\
        On ne l’ouvre pas pour le lire ou pour apprendre.\\
        On l’ouvre pour se souvenir.\\\\
        Derrière chaque coup, chaque pièce déplacée,\\
        il y a Christophe — Krystof126.
    };

    % 4. PLACE THE LOGO IMAGE
    % Calculate the target position: 40% (0.4) of the way down from the top-center
    % to the bottom-center of the safe zone. This centers it horizontally
    % and places it at the correct vertical position within the margins.
    \node[
        inner sep=0pt % No padding around the logo
    ] at ($(SafeZoneTopCenter)!0.33!(SafeZoneBottomCenter)$) {
        % You can change the width of the logo here if needed.
        \includegraphics[width=6cm]{krystof126-logo.png}
    };

\end{tikzpicture}

\end{document}
