\documentclass{article}
\usepackage[margin=1in]{geometry}
\usepackage{lmodern}
\usepackage{xcolor}
\usepackage{xskak}
\usepackage{tikz}
\usetikzlibrary{calc, positioning}
\usepackage{pgfkeys}
\usepackage{etoolbox} % Required for \ifstrequal
\usepackage{xparse}   % For the robust \NewDocumentCommand

%----------------------------------------------------------------------
% Key-value interface for the \boardWithEval command
%----------------------------------------------------------------------
\pgfkeys{
  /boardwithbar/.is family, /boardwithbar,
  default/.style = {
    eval = 0.0,
    board-opts = {},
    bar-position = right,
    bar-gap = 5pt,
    bar-width = 8pt,
    white-color = gray!10,
    black-color = gray!80,
    border-color = black,
    max-eval = 5.0,
    show-score = false,
    score-font = \small
  },
  eval/.store in = \bwbEval,
  board-opts/.store in = \bwbBoardOpts,
  bar-position/.store in = \bwbBarPos,
  bar-gap/.store in = \bwbBarGap,
  bar-width/.store in = \bwbBarWidth,
  white-color/.store in = \bwbWhiteColor,
  black-color/.store in = \bwbBlackColor,
  border-color/.store in = \bwbBorderColor,
  max-eval/.store in = \bwbMaxEval,
  show-score/.store in = \bwbShowScore,
  score-font/.store in = \bwbScoreFont,
  % Styles for positioning the bar, which is more robust than \ifstrequal in a node
  bar on right/.style={anchor=south west, at=(board.south east), xshift=\bwbBarGap},
  bar on left/.style={anchor=south east, at=(board.south west), xshift=-\bwbBarGap},
  % Setup key to robustly handle multi-line arguments.
  setup/.style={#1}
}

%----------------------------------------------------------------------
% The main \boardWithEval command definition (Final Robust Version)
% Usage: \boardWithEval[<options>]{<fen string>}
%----------------------------------------------------------------------
\newsavebox\bwbBox
\NewDocumentCommand{\boardWithEval}{ O{} m }{%
  % Set default values then robustly process user-provided keys from the optional argument #1.
  \pgfkeys{/boardwithbar/.cd, default, setup={#1}}%

  % 1. Robustly build the options list for \chessboard.
  %    \edef ensures that \bwbBoardOpts is fully expanded before being used.
  \edef\bwbTempOpts{setfen={#2}, \bwbBoardOpts}%

  % 2. Render the chessboard into a savebox using the expanded options.
  %    The \expandafter commands ensure \chessboard sees the clean, final option list.
  \sbox\bwbBox{\expandafter\chessboard\expandafter[\bwbTempOpts]}%

  % 3. Pre-calculate all necessary dimensions BEFORE starting the TikZ picture.
  \pgfmathsetmacro{\bwbBoardHeight}{\ht\bwbBox+\dp\bwbBox}%
  \pgfmathsetmacro{\clampedeval}{min(max(\bwbEval, -\bwbMaxEval), \bwbMaxEval)}%
  \pgfmathsetmacro{\whiteheight}{(0.5 + (\clampedeval / \bwbMaxEval) / 2) * \bwbBoardHeight}%

  \leavevmode
  \begin{tikzpicture}[baseline=(board.center)]
    % Place the pre-rendered box into a TikZ node.
    \node (board) [inner sep=0pt, outer sep=0pt] {\usebox\bwbBox};

    % Position and draw the evaluation bar container node using a style.
    \node (evalbar) [
        draw=\bwbBorderColor, rectangle, inner sep=0pt,
        minimum height=\bwbBoardHeight pt, minimum width=\bwbBarWidth,
        /boardwithbar/bar on \bwbBarPos
    ] {};

    % Fill the bar using simple, separate path commands.
    \fill[\bwbWhiteColor] (evalbar.south west) rectangle ++(\bwbBarWidth, \whiteheight pt);
    \fill[\bwbBlackColor] (evalbar.south west) ++(0, \whiteheight pt) rectangle (evalbar.north east);

    % Optionally, display the numerical score.
    \ifstrequal{\bwbShowScore}{true}
    {
      \pgfextra{
        \pgfmathparse{ifthenelse(\bwbEval >= 0, "+", "")}
        \let\sign\pgfmathresult
        \pgfmathprintnumberto[fixed, precision=1]{\bwbEval}{\evalstring}
      }
      \node[anchor=south, inner sep=2pt, font=\bwbScoreFont] at (evalbar.north) {\sign\evalstring};
    }{}
  \end{tikzpicture}%
}

%----------------------------------------------------------------------
% Document Body - Example Usage
%----------------------------------------------------------------------
\begin{document}

\section*{Showcase of the \texttt{\textbackslash boardWithEval} Command}

Here is a sample analysis demonstrating the command at various points in a game.

\subsection*{Position 1: A Balanced Opening (Ruy Lopez)}
After 1. e4 e5 2. Nf3 Nc6 3. Bb5 a6, we reach a standard position in the Ruy Lopez, Morphy Defense. The position is considered dynamically balanced, with a very slight initiative for White.
\begin{center}
\boardWithEval[eval=0.2]{r1bqkbnr/1ppp1ppp/p1n5/1B2p3/4P3/5N2/PPPP1PPP/RNBQK2R w KQkq - 0 4}
\end{center}

\subsection*{Position 2: White's Clear Advantage (Kasparov vs. Topalov, 1999)}
This position is from the famous game between Garry Kasparov and Veselin Topalov. After Kasparov's brilliant rook sacrifice 24. Rxd4!!, White has a powerful attack and a significant advantage.
\begin{center}
\boardWithEval[eval=4.5, board-opts={showmover=true}]{r1b3k1/1p3p1p/p1n1p1p1/8/2Rr1P2/2N2N2/PPP2P1P/2K5 b - - 1 24}
\end{center}

\subsection*{Position 3: Black's Decisive Advantage (Byrne vs. Fischer, 1956)}
From the "Game of the Century," this position arose after Bobby Fischer's stunning move 17... Be6!!. This quiet move unleashes a devastating attack, leaving Black with a completely winning position.
\begin{center}
\boardWithEval[eval=-6.1, max-eval=7.0, board-opts={showmover=true}]{1r2k2r/p1p3pp/2n1bn2/1p2pp2/2P2P2/1PNP1N2/P1Q1P1BP/1R3RK1 b k - 0 18}
\end{center}

\subsection*{Position 4: Customization Showcase}
The command is highly customizable. Here, the bar is placed on the left, its colors are changed to light blue and dark red, it's made wider, and the numerical score is displayed above it.
\begin{center}
% The multi-line options are now safe to use.
\boardWithEval[
  eval=-1.2,
  bar-position=left,
  bar-width=12pt,
  bar-gap=8pt,
  white-color=blue!20,
  black-color=red!60!black,
  border-color=gray,
  show-score=true,
  score-font=\bfseries\small
]{4k3/8/8/8/8/8/8/R3K2R w KQ - 0 1}
\end{center}

\end{document}
